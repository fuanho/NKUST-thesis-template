

\setcounter{secnumdepth}{6}
\setcounter{tocdepth}{6}

% \usepackage{showframe} % 顯示邊界線(不需要請註解掉)

% ----------------------------
%   文字字型設定
% ----------------------------

% xelatex 中文套件設定
\usepackage[SlantFont]{xeCJK} % 中文字體支援(xelatex)
\usepackage[LGR,OT1]{fontenc} % Output font encoding for international characters
\usepackage{palatino} % Use the Palatino font by default
\usepackage{indentfirst} \setlength{\parindent}{2em}
\usepackage{lipsum}
\usepackage{fontspec}
\usepackage[english]{babel} % 外國語系字母支援

\defaultfontfeatures{AutoFakeBold=2.5,AutoFakeSlant=.2}

% 英文字體設定(可自行更改)
\setmainfont[Path=./Fonts/]{DejaVuSerif}

% 英文字家族範例
\renewcommand*{\rmdefault}{lmr}
\renewcommand*{\sfdefault}{lmss}
\renewcommand*{\ttdefault}{lmtt}


% 中文字體設定(教育部標準字體)
\setCJKfamilyfont{kai}[Path=./Fonts/]{edukai-4.0}
\newcommand*{\edukai}{\CJKfamily{kai}}

\setCJKfamilyfont{song}[Path=./Fonts/]{edusong-big5}
\newcommand*{\edusong}{\CJKfamily{song}}

\setCJKmainfont[Path=./Fonts/]{edusong-big5}
\setCJKmonofont[Path=./Fonts/]{edusong-big5}
\setCJKsansfont[Path=./Fonts/]{edusong-big5}


% ----------------------------
%   排版擴充
% ----------------------------
\newcommand\n{\mbox{\qquad}} % 換行指令
\usepackage{xcolor}	% 字體顏色
\usepackage{lettrine}  % 空白間距
\usepackage[autostyle=true]{csquotes} % 引號排版
\usepackage{indentfirst} % 首行縮排
\usepackage{setspace} % 行距
\usepackage{scrbase}
\usepackage{scrhack}
\usepackage{Templates/nkusthesis_layout} % 版面排版

% ----------------------------
%   排序
% ----------------------------
\usepackage{amsmath}

\numberwithin{figure}{section}

\newtheorem{definition}{Definition}
\newtheorem{problem}{Problem}

\newtheorem{theorem}{Theorem}[section]
\newtheorem{lemma}[theorem]{Lemma}
\newtheorem{note}[theorem]{Note}
\newtheorem{corollary}[theorem]{Corollary}
\newtheorem{prop}[theorem]{Proposition}

\newtheorem{thm}{Theorem}%[section]
\newtheorem{lma}{Lemma}%[theorem]
\newtheorem{defi}{Definition}
\newtheorem{rul}{Rule}

% ----------------------------
%   章節
% ----------------------------
\usepackage{titlesec}
\usepackage{titletoc}
\usepackage{zhnumber}
\usepackage{CJKnumb}	% 阿拉伯數字轉中文字

% ----------------------------
%   超連結
% ----------------------------
\usepackage{hyperref}
\usepackage{url}

\hypersetup{pdfpagemode={UseOutlines},
	linktocpage,		% 
	bookmarksopen=true,
	bookmarksopenlevel=0,
	hypertexnames=false,
	colorlinks=true, % Set to false to disable coloring links
	citecolor=red, % The color of citations
	linkcolor=blue, % The color of references to document elements (sections, figures, etc)
	urlcolor=blue, % The color of hyperlinks (URLs)
	pdfstartview={FitV},
	unicode,
	breaklinks=true,
	pdfcreator={NKUST LaTeX Template Groups},
	pdfproducer={NKUST LaTeX template build on XeLaTeX},
	pdftitle={\titletw\ \titleen},						% 文件標題
	pdfsubject={\authorenname, \degreeen\ thesis - \titleen},	% 文件主旨
	pdfauthor={\authorenname \ \authortwname},					% 擁有者
	pdfkeywords={\thesiskeywords},	% 關鍵字
}

\pdfstringdefDisableCommands{ % If there is an explicit linebreak in a section heading (or anything printed to the pdf-bookmarks), it is replaced by a space
	\let\\\space
}

% ----------------------------
%   內文排版
% ----------------------------

% 設定行距為 1.5
\linespread{1.5}

% 章節上面不要留空
\titlespacing*{\chapter}{0pt}{-25pt}{40pt}

% 圖跟表目錄前面不要有空格
\makeatletter
\renewcommand*\l@figure{\@dottedtocline{1}{0em}{2.3em}}% Default: 1.5em/2.3em
\let\l@table\l@figure
\makeatother

%   程式碼區塊
\usepackage{listings}
\usepackage{xcolor}

\definecolor{codegreen}{rgb}{0,0.6,0}
\definecolor{codegray}{rgb}{0.5,0.5,0.5}
\definecolor{codepurple}{rgb}{0.58,0,0.82}
\definecolor{backcolour}{rgb}{0.95,0.95,0.92}
\lstdefinestyle{codestyle}{
	backgroundcolor=\color{backcolour},   
	commentstyle=\color{codegreen},
	keywordstyle=\color{magenta},
	numberstyle=\tiny\color{codegray},
	stringstyle=\color{codepurple},
	basicstyle=\textrm\footnotesize\ttfamily,
	breakatwhitespace=false,         
	breaklines=true,                 
	captionpos=b,                    
	keepspaces=true,                 
	numbers=left,                    
	numbersep=5pt,                  
	showspaces=false,                
	showstringspaces=false,
	showtabs=false,                  
	tabsize=2
}

\lstset{style=codestyle}

%   演算法虛擬碼
\usepackage{algorithm}
\usepackage{algorithmicx}
\usepackage{algpseudocode}

%   小圖
\usepackage{subfigure}

%   原文顯示支援
\usepackage{verbatim}
\usepackage{moreverb}



%   表格
\usepackage{multirow}
\usepackage{makecell}
\usepackage{threeparttable}
\usepackage{longtable}
\usepackage{booktabs}

%   外部PDF導入
\usepackage{pdfpages}

% 曲線繪圖
\usepackage{tikz}

% 資料繪圖
\usepackage{pgfplots}
\usepgfplotslibrary{colorbrewer}
\pgfplotsset{compat=newest,compat/show suggested version=false}
\usepackage[subpreambles]{standalone}

% 繪圖
\usepackage{graphicx} % Required to include images
\graphicspath{{Figures/}{./}} % Specifies where to look for included images

% 流程狀態圖繪圖
\usepackage{amscd}

% 數學符號擴充
\usepackage{amssymb}

% 數學公式擴充
\usepackage{amsthm}

% 符號擴充
\usepackage{siunitx}

% 圖片
\usepackage{eso-pic}
\usepackage{picture}
\usepackage{wallpaper}

% 浮水印
\usepackage[contents=]{background} %如果沒有給 contest空白會自動加上 draft 浮水印字樣

% 浮水印參數定義
\newcommand\NTUSTwatermark {
	\backgroundsetup{
		contents={\includegraphics{\watermarkimage}},
		angle=0,
		opacity=0.2,
		hshift = 0mm,
		vshift = 0mm,
		scale=0.5
	}
}

% 章節
\usepackage{caption}
\captionsetup[centerlast,small,n]{name=caption} % setting caption
\setlength{\captionmargin}{50pt}

% ----------------------------
%   頁首頁尾
% ----------------------------
\usepackage[Lenny]{fncychap}
\pagestyle{empty}

% ----------------------------
%   其他擴充功能
% ----------------------------

%   if-else 擴充
\usepackage{ifthen}

%   variable checked
\usepackage{etoolbox}

% ----------------------------
%   其他定義
% ----------------------------

% Other
\newcommand{\checktoopen}{% New command to move content to the next page which prints to the next odd page if twosided mode is active
\if@openright\cleardoublepage\else\clearpage\fi
}

\newcommand\bhrule{\typeout{--------------------}}
\newcommand\tttypeout[1]{\bhrule\typeout{\space #1}\bhrule}

\newcommand{\HRule}{\rule{\linewidth}{0.5mm}} % New command to make the lines in the title page
\newcommand{\decoRule}{\rule{.8\textwidth}{.4pt}} % New command for a rule to be used under figures

% \renewcommand{}{\vspace*{1pt}} % Reduce the whitespace above a chapter heading

\setcounter{tocdepth}{3} % The depth to which the document sections are printed to the table of contents
\providecommand\addchaptertocentry[1]{%
\addcontentsline{toc}{chapter}{#1}}

% Penalties
\doublehyphendemerits=10000 % No consecutive line hyphens
\brokenpenalty=10000 % No broken words across columns/pages
\widowpenalty=9999 % Almost no widows at bottom of page
\clubpenalty=9999 % Almost no orphans at top of page
\interfootnotelinepenalty=9999 % Almost never break footnotes

% 參考文獻
\newcommand{\fref}[1]{\figurename~\ref{#1}}
\newcommand{\tref}[1]{\tablename~\ref{#1}}
\providecaptionname{german,ngerman,austrian,naustrian}{\equationnamenname}{Formel}
\providecaptionname{american,australian,british,canadian,english,newzealand,UKenglish,USenglish}{\equationnamenname}{Equation}
\newcommand{\eref}[1]{\equationname~\ref{#1}}
\newcommand{\cref}[1]{\chaptername~\ref{#1}}
\newcommand{\sref}[1]{\sectionname~\ref{#1}}
\providecaptionname{german,ngerman,austrian,naustrian}{\sectionname}{Abschnitt}
\providecaptionname{american,australian,british,canadian,english,newzealand,UKenglish,USenglish}{\sectionname}{Section}
\newcommand{\aref}[1]{\appendixname~\ref{#1}}

% ----------------------------
%   導入NKUSThesis版型設定
% ----------------------------
\usepackage{Templates/abstract_template}
\usepackage{Templates/acknowledgements_template}
\usepackage{Templates/reference_template}
\usepackage{Templates/chapter_template}

%   條列式清單
\usepackage{enumerate}
\usepackage{enumitem}

% -----------------------------------------------------------------------------------------------


